\documentclass{article}

\usepackage[utf8]{inputenc}
\usepackage{amsmath}
\usepackage{graphicx}
\usepackage{hyperref}

\title{F-ATP Synthase}
\author{Emelianov Artem \\ Lomonosov Moscow State University \\ yaz008.yaz008@yandex.ru}
\date{27.10.2024}

\begin{document}

\maketitle

\section{Introduction}

ATP synthase is a crucial enzyme that plays a significant role in cellular energy production. It is located in the inner mitochondrial membrane of eukaryotic cells or in the plasma membrane of prokaryotes. ATP synthase harnesses the energy from a proton gradient generated by the electron transport chain to synthesize ATP from ADP and inorganic phosphate. 

\section{Enzyme Structure}

ATP synthase structure varies across different organisms, but the main parts remain similar. The enzyme is generally composed of two main domains: the \textit{$F_1$} and \textit{$F_0$} domains.

\subsection{Subunits}

In \textit{E. coli}, the \textit{$F_1$} domain consists of five distinct subunits: $\alpha$, $\beta$, $\gamma$, $\delta$, and $\epsilon$. The stoichiometry of these subunits is represented as $3\alpha : 3\beta : 1\gamma : 1\delta : 1\epsilon$. The \textit{$F_0$} domain, which is responsible for proton translocation across the membrane, comprises three additional subunits: $a$, $b$, and $c$. The ratio of these subunits in \textit{E. coli} is $1a : 2b : 10c$.

In contrast, the mitochondrial \textit{$F_1$} factor exhibits a similar composition but includes an additional subunit known as OSCP (oligomycin sensitivity conferring protein). The stoichiometry for mitochondrial ATP synthase can be described as $3\alpha : 3\beta : 1\gamma : 1\delta : 1\epsilon : 1\text{OSCP}$.

In animal cells, the structure becomes even more complex. The generally accepted stoichiometric coefficients for ATP synthase in these cells are $3\alpha : 3\beta : 1\gamma : 1\delta : 1\epsilon : 1\text{OSCP} : 1a : 1b : 8c$.

\subsection{3D Structure}

\subsubsection{F-One}

The \textit{$F_1$} sector has been extensively studied, and its three-dimensional structure has been determined with high resolution (2.4 Å) through X-ray crystallography. The \textit{$F_1$} complex is predominantly composed of three types of subunits: $3\alpha : 3\beta : 1\gamma$. The central part of the enzyme consists of long $\alpha$-helical segments of the $\gamma$ subunit, surrounded alternately by $\alpha$ and $\beta$ subunits.

The $\alpha$ and $\beta$ subunits each contain a nucleotide-binding site, while the $\gamma$ subunit plays a crucial structural role without binding nucleotides under physiological conditions. The $\gamma$ subunit extends asymmetrically from the globule's center, forming a "stalk" that connects \textit{$F_1$} to the \textit{$F_0$} domain.

\subsubsection{F-Zero}

The \textit{$F_0$} sector is integral to proton translocation across the membrane. It consists of a, b, and c types of subunits.

The c subunits forms an oligomeric ring structure, in which the actual number of elements varies among different organisms. For instance, \textit{E. coli} typically contains 10 c subunits. Each c subunit comprises two transmembrane $\alpha$-helices connected by a loop rich in polar amino acids (which is crucial for proton transport).

The a subunit is hydrophobic and forms several transmembrane $\alpha$-helices (that likely contribute to forming the proton-conducting pathway).

In contrast, the b subunit appears more hydrophilic and helps stabilize the assembly of \textit{$F_1$} and \textit{$F_0$} by forming peripheral connections (peripheral leg) of ATP synthase.

\section{Mechanism}

ATP synthase is situated in the inner mitochondrial membrane (or thylakoid membrane in chloroplasts) and exploits a proton gradient established by the electron transport chain. Protons ($H^+$ ions) flow back into the mitochondrial matrix through the \textit{$F_0$} component of ATP synthase.

The \textit{$F_0$} portion is embedded in the membrane and consists of multiple subunits, including c subunits that form a rotating ring. As protons enter this ring, they induce rotation of the c subunits. Each proton that passes through causes a conformational change that facilitates the rotation of the c-ring.

The rotation of the c-ring is transmitted to the \textit{$F_1$} component via the central stalk (composed of the $\gamma$ subunit). This rotation causes conformational changes in the $\beta$ subunits of \textit{$F_1$}, which are responsible for catalyzing ATP synthesis. The \textit{$F_1$} component has three $\beta$ subunits, each capable of binding ADP and inorganic phosphate to form ATP.

As the $\beta$ subunits undergo conformational changes during each turn ("loose," "tight," and "open" states), they sequentially bind ADP and inorganiv phosphate, convert them into ATP, and release it into the mitochondrial matrix. This process is driven by the mechanical energy derived from proton flow.

The activity of ATP synthase can be regulated by various factors, including ADP availability and the proton motive force. When ADP levels are high, ATP synthase operates more actively to generate ATP.

\section{Variance}

\subsection{Evolution}

All rotary ion-translocating ATPases share a common evolutionary origin, suggesting that they evolved from a common ancestor that possessed the basic structural features necessary for ATP synthesis and ion transport. This evolutionary relationship is supported by the conservation of key subunits involved in catalysis across all types of ATP synthases.

\subsection{Structural Differences}

\subsubsection{F-Type}

F-type ATP synthases are characterized by a relatively simple structure comprising a hydrophilic \textit{$F_1$} subcomplex and a hydrophobic \textit{$F_0$} subcomplex. The \textit{$F_1$} portion contains a hexameric ring formed by alternating $\alpha$ and $\beta$ subunits, which houses the nucleotide-binding sites. The central stalk, composed of the $\gamma$ subunit, connects the \textit{$F_1$} and \textit{$F_10$} complexes, facilitating rotational movement during ATP synthesis. Notably, F-type ATP synthases from different organisms (e.g., bacteria and chloroplasts) exhibit minimal structural differences, maintaining a conserved assembly of subunits that is crucial for their catalytic function. Mitochondrial F-type ATP synthases, however, include an additional $\epsilon$ subunit in their central stalk, enhancing their complexity compared to bacterial counterparts.

\subsubsection{V-Type}

V-type ATP synthases differ significantly in structure from F-type enzymes. They possess multiple peripheral stalks - typically three - each consisting of two subunits (e and g). This structural variation allows for more complex interactions within eukaryotic cells, where V-type ATP synthases primarily function as proton pumps rather than synthesizers of ATP. The central stalk in V-type enzymes contains additional unique subunits that facilitate the attachment of the functional d subunit, which is not present in F-type ATP synthases. The presence of a hydrophilic domain in the a subunit of V-type ATP synthases resembles a "collar" that helps in the attachment of peripheral stalks.

\subsubsection{A-Type}

A-type ATP synthases share some similarities with both F- and V-types but are primarily found in archaea and certain bacteria. They typically feature two peripheral stalks and a central stalk that connects to the catalytic sites. The structural components of A-type enzymes are thought to have evolved from ancestral forms shared with V-type ATP synthases, which indicates a close evolutionary relationship.

\subsection{Functional Differences}

F-type ATP synthases primarily function as ATP synthases, catalyzing the synthesis of ATP from ADP and inorganic phosphate (Pi) using the energy derived from a proton electrochemical gradient across the membrane. However, they can also operate in reverse under certain conditions, hydrolyzing ATP to maintain proton gradients when necessary.

In contrast, V-type ATP synthases are primarily proton pumps and do not synthesize ATP directly. Instead, they utilize ATP hydrolysis to transport protons across membranes, generating a proton electrochemical gradient.

A-type ATP synthases exhibit functional versatility similar to F-type enzymes but are primarily found in archaea and some bacteria. They can function both as ATP synthases and as ATP-dependent proton pumps, depending on the environmental conditions.

\section{ADP Negative Regulation}

\subsection{Overview}

ATP synthase catalyzes the synthesis of ATP from ADP and inorganic phosphate, utilizing the proton motive force generated by electron transport chains, but it can reverse its function and act as an ATP-dependent proton pump.

The mechanism of ADP inhibition of ATP synthase primarily involves non-competitive inhibition by ADP when it binds to the catalytic site of the enzyme without the presence of inorganic phosphate.

\subsection{Process}

When ADP binds to the catalytic site of $F_0F_1$-complex, it can induce conformational changes in the enzyme. This binding occurs in the absence of Pi.

The conformational changes caused by ADP binding can lead to a situation where the release of ADP from the catalytic site is hindered. This effectively inactivates the enzyme, preventing further ATP synthesis and leading to a state where ATP hydrolysis may occur instead.

The release of tightly bound ADP and reactivation of ATP synthase can occur when there is a sufficient proton motive force. However, the threshold for reactivation is often higher than that required for ATP synthesis, indicating that under low energy conditions, the enzyme remains inactive despite potential proton gradients.

Energization of the membrane enhances the affinity of the catalytic site for inorganic phosphate, thereby reducing the likelihood of ADP being present without it. This mechanism helps prevent the transition into an ADP-inactivated state.

Certain compounds such as alcohols, sulfites, and detergents can weaken ADP inhibition and enhance ATPase activity. These compounds may interact with ATP synthase in ways that stabilize its active form or facilitate the release of ADP.

\subsection{Regulatory Proteins}

\begin{itemize}
    \item  \textbf{IF1 Protein}: In mitochondria, IF1 inhibits ATP hydrolysis under low pH conditions.
    \item \textbf{$\epsilon$ Subunit}: In chloroplasts and some bacteria, this subunit can suppress ATP hydrolysis by blocking the rotation necessary for catalysis.
    \item \textbf{$\zeta$ Protein}: Found in Paracoccus denitrificans, it also inhibits ATPase activity similarly to IF1.
\end{itemize}

\section{Subunit Epsilon}



\section{Conformational Transitions}

\begin{thebibliography}{9}
    \bibitem{ref1} \emph{ATP Synthase}, Bioenergetica, 2011.
    \bibitem{ref2} V.M. Zubareva, A.S. Lapashina, T.E. Shugaeva, A.V. Litvin, B.A. Feniouk, \emph{Rotary Ion Translocating ATPases/ATP Synthases: Diversity, Common Features, and Differences}, Biochemistry, 2020.
    \bibitem{ref3} A.S. Lapashina, B.A. Feniouk, \emph{ADP Inhibition of H+-F0F1-ATP Synthase}, Biochemistry, 2018.
    \bibitem{ref4} Boris A. Feniouk a, Toshiharu Suzuki, Masasuke Yoshida, \emph{The role of subunit epsilon in the catalysis and regulation of FOF1-ATP synthase}, Biochimica et Biophysica Acta (BBA) - Bioenergetics, 2006.
    \bibitem{ref5} Boris A. Feniouk, Yasuyuki Kato-Yamada, Masasuke Yoshida, and Toshiharu Suzuki, \emph{Conformational Transitions of Subunit Epsilon in ATP Synthase from Thermophilic Bacillus PS3}, Biophysical Journal, 2010.
\end{thebibliography}

\end{document}