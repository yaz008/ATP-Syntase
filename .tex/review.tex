\documentclass{article}

\usepackage[utf8]{inputenc}
\usepackage{amsmath}
\usepackage{graphicx}
\usepackage{hyperref}

\title{F-ATP Synthase}
\author{Emelianov Artem \\ Lomonosov Moscow State University \\ yaz008.yaz008@yandex.ru}
\date{27.10.2024}

\begin{document}

\maketitle

\section{Introduction}

ATP synthase is a crucial enzyme that plays a significant role in cellular energy production. It is located in the inner mitochondrial membrane of eukaryotic cells and in the plasma membrane of prokaryotes. ATP synthase harnesses the energy from a proton gradient generated by the electron transport chain to synthesize ATP from ADP and inorganic phosphate. 

\section{Enzyme Structure}

ATP synthase structure varies across different organisms, but the main parts remain similar. The enzyme is generally composed of two main domains: the \textit{$F_1$} and \textit{$F_0$}.

\subsection{Subunits}

In \textit{E. coli}, the \textit{$F_1$} domain consists of five distinct subunits: $\alpha$, $\beta$, $\gamma$, $\delta$, and $\epsilon$. The stoichiometric ratio of $3\alpha : 3\beta : 1\gamma : 1\delta : 1\epsilon$. The \textit{$F_0$} domain, which is responsible for proton translocation across the membrane, comprises three additional subunits: $a$, $b$, and $c$. The ratio of these subunits in \textit{E. coli} is $1a : 2b : 10c$.

In contrast, the mitochondrial \textit{$F_1$} factor exhibits a similar composition but includes an additional subunit known as OSCP (oligomycin sensitivity conferring protein). The stoichiometry for mitochondrial ATP synthase can be described as $3\alpha : 3\beta : 1\gamma : 1\delta : 1\epsilon : 1\text{OSCP}$.

In animal cells, the structure becomes a bit different. The generally accepted stoichiometric coefficients for ATP synthase in these cells are $3\alpha : 3\beta : 1\gamma : 1\delta : 1\epsilon : 1\text{OSCP} : 1a : 1b : 8c$.

\subsection{3D Structure}

\subsubsection{F-One}

The \textit{$F_1$} sector has been extensively studied, and its three-dimensional structure has been determined with high resolution (2.4 Å) through X-ray crystallography. The \textit{$F_1$} complex is predominantly composed of three types of subunits: $3\alpha : 3\beta : 1\gamma$. The central part of the enzyme consists of long $\alpha$-helical segments of the $\gamma$ subunit, surrounded alternately by $\alpha$ and $\beta$ subunits.

The $\alpha$ and $\beta$ subunits each contain a nucleotide-binding site, while the $\gamma$ subunit plays a crucial structural role without binding nucleotides under physiological conditions. The $\gamma$ subunit extends asymmetrically from the globule's center, forming a "stalk" that connects \textit{$F_1$} to the \textit{$F_0$} domain. The $\epsilon$ subunit resides closer to the \textit{$F_0$} sector and is involved in regulation.

\subsubsection{F-Zero}

The \textit{$F_0$} sector is integral to proton translocation across the membrane. It consists of a, b, and c types of subunits.

The c subunits forms an oligomeric ring structure, in which the actual number of elements varies among different organisms. For instance, \textit{E. coli} typically contains 10 c subunits. Each c subunit comprises two transmembrane $\alpha$-helices connected by a loop rich in polar amino acids (which is crucial for proton transport).

The a subunit is hydrophobic and forms several transmembrane $\alpha$-helices (that likely contribute to forming the proton-conducting pathway).

In contrast, the b subunit appears more hydrophilic and helps stabilize the assembly of \textit{$F_1$} and \textit{$F_0$} by forming "peripheral leg" of ATP synthase.

\section{Mechanism}

ATP synthase is integrated in the inner mitochondrial membrane (or thylakoid membrane in chloroplasts) and utilizes a proton gradient established by the electron transport chain. Protons ($H^+$) flow back into the mitochondrial matrix through the \textit{$F_0$} sector of ATP synthase.

The \textit{$F_0$} portion is embedded in the membrane and consists of multiple subunits, including c subunits that form a rotating ring. As protons enter this ring, they induce rotation of the c subunits. Each proton that passes through causes a conformational change that facilitates the rotation of the c-ring.

The rotation of the c-ring is transmitted to the \textit{$F_1$} component via the central stalk (composed of the $\gamma$ subunit). This rotation causes conformational changes in the $\beta$ subunits of \textit{$F_1$}, which are responsible for catalyzing ATP synthesis. The \textit{$F_1$} component has three $\beta$ subunits, each capable of binding ADP and inorganic phosphate to form ATP.

As the $\beta$ subunits undergo conformational changes during each turn ("loose," "tight," and "open" states), they sequentially bind ADP and inorganiv phosphate, convert them into ATP, and release it into the mitochondrial matrix. This process is driven by the mechanical energy (torsion of the $\gamma$ subunit) derived from proton flow.

The activity of ATP synthase can be regulated by various factors, including ADP availability and the proton motive force. When ADP levels are high, \textit{$F_1F_0$}  complex operates more actively to generate ATP.

\section{Variance of ATP Synthases}

\subsection{Evolution}

All rotary ion-translocating ATPases share a common evolutionary origin, suggesting that they evolved from a common ancestor that possessed the basic structural features necessary for ATP synthesis and ion transport. This evolutionary relationship is supported by the conservation of key subunits involved in catalysis across all types of ATP synthases.

\subsection{Structural Differences}

\subsubsection{F-Type}

F-type ATP synthases are characterized by a relatively simple structure comprising a hydrophilic \textit{$F_1$} subcomplex and a hydrophobic \textit{$F_0$} subcomplex. The \textit{$F_1$} portion contains a hexameric ring formed by alternating $\alpha$ and $\beta$ subunits, where the nucleotide-binding sites reside. The central stalk, composed of the $\gamma$ subunit, connects the \textit{$F_1$} and \textit{$F_0$} complexes, facilitating rotational movement during ATP synthesis. Notably, F-type ATP synthases from different organisms (e.g., bacteria and chloroplasts) exhibit small structural differences, maintaining a conserved assembly of subunits that is crucial for their catalytic function. Mitochondrial F-type ATP synthases, however, include an additional $\epsilon$ subunit in their central stalk, raising their complexity compared to bacterial counterparts.

\subsubsection{V-Type}

V-type ATP synthases differ significantly in structure from F-type enzymes. They possess multiple peripheral stalks - typically three - each consisting of two subunits (e and g). This structural variation allows for more complex interactions within eukaryotic cells, where V-type ATP synthases primarily function as proton pumps rather than synthesizers of ATP. The central stalk in V-type enzymes contains additional unique subunits that aid the attachment of the functional d subunit, which is not present in F-type ATP synthases. The presence of a hydrophilic domain in the a subunit of V-type ATP synthases helps in the attachment of peripheral stalks.

\subsubsection{A-Type}

A-type ATP synthases share some similarities with both F- and V-types but are primarily found in archaea and certain bacteria. They typically have two peripheral stalks and a central stalk that connects to the catalytic sites. The structural components of A-type enzymes are thought to have evolved from ancestral forms shared with V-type ATP synthases (a close evolutionary relationship).

\subsection{Functional Differences}

F-type ATP synthases primarily function as ATP synthases, catalyzing the synthesis of ATP from ADP and inorganic phosphate using the energy derived from a proton electrochemical gradient across the membrane. However, they can also operate in reverse under certain conditions, hydrolyzing ATP to maintain proton gradients when necessary.

In contrast, V-type ATP synthases are primarily proton pumps and do not synthesize ATP directly. Instead, they utilize ATP hydrolysis to transport protons across membranes, generating a proton electrochemical gradient.

A-type ATP synthases show functional versatility similar to F-type enzymes (but thay are primarily found in archaea and some bacteria). They can function both as ATP synthases and as ATP-dependent proton pumps, depending on the environmental conditions.

\section{ADP Regulation}

\subsection{Overview}

ATP synthase catalyzes the synthesis of ATP from ADP and inorganic phosphate, utilizing the proton motive force generated by electron transport chains, but it can reverse its function and act as an ATP-dependent proton pump.

The mechanism of ADP inhibition of ATP synthase primarily involves non-competitive inhibition by ADP when it binds to the catalytic site of the enzyme without the presence of inorganic phosphate.

\subsection{Process}

When ADP binds to the catalytic site of $F_0F_1$-complex, it can induce conformational changes in the enzyme. This binding occurs in the absence of inorganic phosphate.

The conformational changes caused by ADP binding can lead to a situation where the release of ADP from the catalytic site is hindered. This effectively inactivates the enzyme, preventing further ATP synthesis and leading to a state where ATP hydrolysis may occur instead.

The release of tightly bound ADP and reactivation of ATP synthase can occur when there is a sufficient proton motive force. However, the threshold for reactivation is often higher than that required for ATP synthesis, indicating that under low energy conditions the enzyme remains inactive despite potential proton gradients.

Energization of the membrane enhances the affinity of the catalytic site for inorganic phosphate, thereby reducing the likelihood of ADP being present without it. This mechanism helps prevent the transition into an ADP-inactivated state.

Certain compounds such as alcohols, sulfites, and detergents can weaken ADP inhibition and enhance ATPase activity. These compounds may interact with ATP synthase in ways that stabilize its active form or facilitate the release of ADP.

\subsection{Regulatory Proteins}

\begin{itemize}
    \item  \textbf{IF1 Protein}: In mitochondria, IF1 inhibits ATP hydrolysis under low pH conditions.
    \item \textbf{$\epsilon$ Subunit}: In chloroplasts and some bacteria, this subunit can suppress ATP hydrolysis by blocking the rotation necessary for catalysis.
\end{itemize}

\section{Subunit Epsilon}

\subsection{Structural context}

The epsilon subunit is part of the \textit{$F_1$} sector of ATP synthase, which is responsible for ATP synthesis. Its position and structure allow it to interact closely with other subunits, particularly the alpha and beta subunits that form the catalytic sites.

\subsection{F0F1 Complex Assembly}

The assembly of ATP synthase is a multi-step process that requires precise coordination among its subunits. The $\epsilon$ subunit contributes to this process by stabilizing intermediate forms of the complex as it assembles.
The N-terminal $\beta$-sandwich part of the $\epsilon$ subunit plays crucial role in \textit{$F_0F_1$} complex assembly in different organisms.

\subsection{Inhibition of ATP Hydrolysis}

One of the primary regulatory functions of the $\epsilon$ subunit is to inhibit ATP hydrolysis under certain conditions. This inhibition is crucial for preventing energy wastage when ATP levels are adequate. The epsilon subunit achieves this by:

\begin{itemize}
    \item  \textbf{Conformational Changes}: The $\epsilon$ subunit can undergo conformational shifts that alter its interaction with the catalytic sites. When ATP levels are high, these shifts can stabilize an extended conformation that inhibits hydrolysis.
    \item  \textbf{Binding Dynamics}: The binding of ADP and inorganic phosphate can influence the state of the $\epsilon$ subunit, promoting an extended conformation that favors ATP synthesis while preventing hydrolysis.
\end{itemize}

\subsection{Regulation via Proton Motive Force}

The activity of ATP synthase, including that of the $\epsilon$ subunit, is influenced by the proton motive force. It drives protons through the \textit{$F_0$} sector, causing rotation that impacts the \textit{$F_1$} sector's conformation ($\alpha$ + $\beta$ trimer in particular). A strong proton motive force may activate ATP synthesis, while a weaker force could trigger its hydrolysis.

\subsection{Role in Catalytic Mechanism}

The $\epsilon$ subunit assists in the proper positioning and orientation of nucleotides at the catalytic sites, enhancing ATP synthesis efficiency.

The interaction between the $\epsilon$ subunit and other components facilitates the rotational coupling necessary for ATP production. This coupling is essential for converting mechanical (rotational and torsional) energy obtained from proton flow into chemical energy stored in ATP.

\subsection{Interaction with Other Subunits}

The $\epsilon$ subunit has specific binding sites that facilitate its interaction with both alpha and beta subunits. These interactions are crucial for maintaining the structural integrity of the ATP synthase complex during ATP synthesis.

Upon binding to the alpha and beta subunits, the $\epsilon$ subunit induces conformational changes that are necessary for ATP synthesis. These changes are linked to the rotation of the gamma subunit. The $\epsilon$ subunit acts as a stabilizer during this process, ensuring that the catalytic sites remain properly aligned.

The $\epsilon$ subunit forms direct contacts with specific residues on the alpha subunits, which helps stabilize their conformation during ATP synthesis. The interaction with beta subunits is more complex, it involves multiple points of contact. They ensure structural stability. The $\epsilon$ subunit's binding can modulate how efficiently beta subunits can transition between different states.

\section{Conformational Transitions}

\subsection{Inistial State}

The initial state of the epsilon subunit of ATP synthase involves several key characteristics.

\begin{itemize}
    \item  \textbf{Compact Structure}: In its initial state, the $\epsilon$ subunit is typically in a compact conformation. This compactness is crucial for its interaction with other subunits of ATP synthase, particularly the catalytic core (the $\alpha$ and $\beta$ subunits).
    \item \textbf{Positioning}: The $\epsilon$ subunit is positioned within the rotor complex, which is part of the larger ATP synthase structure. Its placement is strategic for facilitating the necessary conformational changes during ATP synthesis.
    \item \textbf{Binding Sites}: The $\epsilon$ subunit contains specific binding sites that allow it to interact with the c-ring of the rotor. This interaction is essential for transmitting rotational movements that are generated by proton flow through the membrane.
    \item \textbf{Stability}: The initial state of the $\epsilon$ subunit is stabilized by interactions with adjacent subunits, ensuring that it remains in place during the initial phases of ATP synthesis.
\end{itemize}

\subsection{Transition to Active State}

Upon proton flow through the c-ring, a conformational change occurs.

\begin{itemize}
    \item  \textbf{Initial Trigger}: The transition to the active state is initiated by the rotation of the c-ring, which is driven by proton flow through the membrane. This rotation applies mechanical force on the $\epsilon$ subunit, causing it to undergo conformational changes.
    \item  \textbf{Rotation-Induced Movement}: This movement aligns the $\epsilon$ subunit with the catalytic sites of the $\alpha$ and $\beta$ subunits, facilitating effective interaction.
    \item  \textbf{Opening of Binding Sites}: During this transition, specific binding sites on the $\epsilon$ subunit become more accessible.
    \item \textbf{Formation of Active Conformation}: The $\epsilon$ subunit takes an extended in its active state. This conformation allows it to effectively transmit mechanical energy from the c-ring rotation to the catalytic core, promoting ATP formation.
\end{itemize}

\subsection{Active State}

\begin{itemize}
    \item \textbf{Extended Conformation}: In its active state, the $\epsilon$ subunit adopts a more elongated conformation compared to its initial state. This structural change is crucial for its interactions with other components of ATP synthase, particularly the catalytic sites of the $\alpha$ and $\beta$ subunits.
    \item \textbf{Dynamic Interactions}: The active $\epsilon$ subunit is capable of dynamic interactions with the c-ring and the catalytic core. These interactions are essential for transmitting mechanical energy derived from proton flow into chemical energy in the form of ATP.
    \item \textbf{Facilitation of ATP Synthesis}: The active conformation of the $\epsilon$ subunit allows it to effectively couple rotational movements from the c-ring to the catalytic sites. This coupling is critical for synthesizing ATP from ADP and inorganic phosphate (Pi).
\end{itemize}

\subsection{Transition to Initial State}

After ATP synthesis, the $\epsilon$ subunit must return to its original conformation:

\begin{itemize}
    \item  \textbf{Loss of Mechanical Forces}: The transition back to the initial state is primarily triggered by the cessation of mechanical forces that were applied during ATP synthesis. As proton flow decreases or stops, the c-ring rotation slows down, leading to a reduction in the torque exerted on the $\epsilon$ subunit.
    \item \textbf{Reversion of Conformational Changes}: In response to the loss of mechanical input, the $\epsilon$ subunit undergoes a series of conformational changes that revert it to its compact structure. This reversion resets the enzyme for another catalytic cycle.
    \item \textbf{Compact Conformation Restoration}: The $\epsilon$ subunit returns to a more compact conformation, which stabilizes the interaction with other subunits. This compact state allows it to prepare for next rotations.
    \item \textbf{Alignment with Catalytic Sites}: During this transition, the $\epsilon$ subunit must realign itself with the catalytic sites of the $\alpha$ and $\beta$ subunits. This alignment is crucial for ensuring that it can participate in following cycles of ATP synthesis.
\end{itemize}

\begin{thebibliography}{9}
    \bibitem{ref1} \emph{ATP Synthase}, Bioenergetica, 2011.
    \bibitem{ref2} V.M. Zubareva, A.S. Lapashina, T.E. Shugaeva, A.V. Litvin, B.A. Feniouk, \emph{Rotary Ion Translocating ATPases/ATP Synthases: Diversity, Common Features, and Differences}, Biochemistry, 2020.
    \bibitem{ref3} A.S. Lapashina, B.A. Feniouk, \emph{ADP Inhibition of H+-F0F1-ATP Synthase}, Biochemistry, 2018.
    \bibitem{ref4} Boris A. Feniouk a, Toshiharu Suzuki, Masasuke Yoshida, \emph{The role of subunit epsilon in the catalysis and regulation of FOF1-ATP synthase}, Biochimica et Biophysica Acta (BBA) - Bioenergetics, 2006.
    \bibitem{ref5} Boris A. Feniouk, Yasuyuki Kato-Yamada, Masasuke Yoshida, and Toshiharu Suzuki, \emph{Conformational Transitions of Subunit Epsilon in ATP Synthase from Thermophilic Bacillus PS3}, Biophysical Journal, 2010.
\end{thebibliography}

\end{document}